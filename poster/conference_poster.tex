\documentclass[a0,portrait]{hogent-poster}

% Info over de opleiding
\course{Bachelorproef}
\studyprogramme{Toegepaste Informatica}
\academicyear{2024-2025}
\institution{Hogeschool Gent, Valentin Vaerwyckweg 1, 9000 Gent}

% Info over de bachelorproef
\title{Alternatief voor Dropbox bij AZL: \textit{DigitalOcean Spaces} als cloudopslagoplossing}
\subtitle{Een onderzoek naar efficiënte en veilige bestandsdeling in sportverenigingen}
\author{Luka Deserranno}
\email{luka.deserranno@student.hogent.be}
\supervisor{Sonia VanderMeersch}
\cosupervisor{Thomas Aelbrecht}
\keywords{cloudopslag, DigitalOcean Spaces, toegangsbeheer, Angular, sportvereniging}
\projectrepo{https://github.com/thomasaelbrecht/azlapp-service}
\projectrepo{https://github.com/thomasaelbrecht/azlapp-web}

\begin{document}
\maketitle

\begin{abstract}
Aquarius Zwemclub Lebbeke (AZL) gebruikt Dropbox voor documentdeling, maar ervaart beperkingen zoals verouderde toegangsrechten en slechte integratie. Deze bachelorproef onderzoekt een alternatief dat beter integreert in de bestaande infrastructuur en gebruiksvriendelijk is voor lesgevers. \textbf{DigitalOcean Spaces} werd geselecteerd en succesvol getest via een proof of concept.
\end{abstract}

\begin{multicols}{2}

\section*{Inleiding}
AZL werkt met een Angular-webapplicatie en WordPress-website. Dropbox sluit daar niet goed op aan. Gebrek aan automatische toegangsbeheer en gebruiksvriendelijkheid belemmeren de werking.

\section*{Probleemstelling}
\begin{itemize}
  \item Geen centrale gebruikersauthenticatie in Dropbox
  \item Beperkte integratie met bestaande Angular-app
  \item Moeilijke toegang voor lesgevers zonder account
\end{itemize}

\section*{Onderzoeksvraag}
Welke cloudopslagoplossing biedt AZL de beste combinatie van:
\begin{itemize}
  \item gebruiksvriendelijkheid,
  \item kostenefficiëntie,
  \item integratie met bestaande systemen,
  \item veilig toegangsbeheer?
\end{itemize}

\section*{Methodologie}
\begin{enumerate}
  \item Literatuurstudie en longlist (Dropbox, S3, Azure, Google, Nextcloud)
  \item Functionele en niet-functionele vereisten definiëren
  \item Shortlist opstellen op basis van requirements
  \item Proof of Concept met DigitalOcean Spaces: integratie en evaluatie
\end{enumerate}

\section*{Resultaten}
\textbf{DigitalOcean Spaces} biedt:
\begin{itemize}
  \item Volledige integratie met bestaande webapp via S3-compatibele API
  \item Backendgestuurde toegang met tijdelijke presigned URLs
  \item Voorspelbaar prijsmodel en snelle implementatie
  \item Geen extra accounts nodig voor lesgevers
\end{itemize}

\begin{center}
  \captionsetup{type=figure}
  \includegraphics[width=0.9\linewidth]{digitalocean-integratie.png} % <-- voeg afbeelding toe
  \captionof{figure}{Schematische integratie van DigitalOcean Spaces met AZL-applicatie}
\end{center}

\section*{Conclusie}
DigitalOcean Spaces voldoet aan de technische en gebruiksvriendelijke eisen van AZL. Het systeem vereenvoudigt toegangsbeheer, is betaalbaar en sluit naadloos aan bij de bestaande infrastructuur.

\section*{Aanbevelingen}
\begin{itemize}
  \item Volledige migratie naar DigitalOcean Spaces
  \item Training voor lesgevers en beheerders
  \item Onderzoek naar mobiele toegang en schaalbaarheid
\end{itemize}

\end{multicols}
\end{document}
