\documentclass[a0,portrait]{hogent-poster}

\course{Bachelorproef}
\studyprogramme{Toegepaste Informatica}
\academicyear{2024–2025}
\institution{Hogeschool Gent, Valentin Vaerwyckweg 1, 9000 Gent}

\title{Digitale Documentopslag voor AZL met DigitalOcean Spaces}
\subtitle{Een Proof of Concept voor veilige en geïntegreerde cloudopslag}
\author{Luka Deserranno}
\email{luka.deserranno@student.hogent.be}
\supervisor{Sonia VanderMeersch}
\cosupervisor{Thomas Aelbrecht (Aquarius Zwemclub Lebbeke)}

\specialisation{Systeem- en Netwerkbeheer}
\keywords{Cloudopslag, DigitalOcean Spaces, Angular, Node.js, Toegangsbeheer}

\begin{document}
\maketitle

\begin{abstract}
Aquarius Zwemclub Lebbeke gebruikte Dropbox om lesmateriaal te delen, maar dit voldeed niet langer aan de vereisten op vlak van toegangsbeheer en integratie. Deze bachelorproef onderzoekt alternatieven en bouwt een werkend prototype met DigitalOcean Spaces, dat veilig, betaalbaar en goed integreerbaar is met AZL’s bestaande infrastructuur.
\end{abstract}

\begin{multicols}{2}

\section{Introductie}
De bestaande Dropbox-omgeving van AZL bood onvoldoende controle over toegangsrechten en geen integratie met de Angular/Node.js webapplicatie. Daarom werd onderzocht welke cloudopslagdienst beter zou passen, met focus op:

\begin{itemize}
  \item Fijnmazig toegangsbeheer
  \item Gebruikersvriendelijkheid
  \item Kostenefficiëntie
  \item Integratiemogelijkheden
\end{itemize}

\section{Onderzoeksvraag}
\textit{Hoe kan AZL documenten veilig en flexibel delen binnen hun bestaande IT-infrastructuur met een schaalbare cloudopslagoplossing?}

\section{Methode}
\begin{itemize}
  \item Analyse van vereisten via MoSCoW-methode
  \item Vergelijking van AWS S3, Nextcloud en DigitalOcean Spaces
  \item Selectie en implementatie van DigitalOcean Spaces
  \item Ontwikkeling van PoC met Angular frontend en Node.js backend
\end{itemize}

\section{Proof of Concept}
\textbf{Toegangsbeheer:}
\begin{itemize}
  \item Eigen permissiesysteem op mappen/bestanden
  \item Fijnmazige controle op basis van gebruikersgroepen
\end{itemize}

\textbf{Backend-integratie:}
\begin{itemize}
  \item AWS SDK voor DigitalOcean Spaces
  \item Pre-signed URLs voor veilige uploads/downloads
  \item Middleware die toegang valideert per route
\end{itemize}

\textbf{Frontendcomponent:}
\begin{itemize}
  \item Angular uploadercomponent met drag-and-drop (ngx-dropzone)
  \item Upload voortgang, routebescherming en feedback
\end{itemize}

\section{Figuur}
\begin{center}
  \captionsetup{type=figure}
  %\includegraphics[width=0.9\linewidth]{digitalocean_architectuur.png} % voeg figuur toe
  \captionof{figure}{Architectuur van de geïntegreerde cloudopslagoplossing voor AZL}
\end{center}

\section{Resultaten}
\begin{itemize}
  \item Oplossing voldoet aan alle 'Must Have'-vereisten
  \item Gebruiksvriendelijke integratie zonder extra accounts
  \item Feedback van testgebruikers was overwegend positief
  \item Lage beheerslast en voorspelbare kosten (\$5/maand)
\end{itemize}

\section{Conclusie}
DigitalOcean Spaces is een robuuste en betaalbare cloudopslagoplossing voor sportclubs zoals AZL. De PoC toont aan dat een veilige en gebruiksvriendelijke integratie met Angular en Node.js haalbaar is, zelfs met beperkt budget.

\section{Toekomstig onderzoek}
\begin{itemize}
  \item Ondersteuning voor versiebestanden
  \item Geavanceerde logging en audit trails
  \item Onderzoek naar federatieve toegang met andere verenigingen
\end{itemize}

\end{multicols}
\end{document}
