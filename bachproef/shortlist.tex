\chapter{\IfLanguageName{dutch}{Shortlist en Selectie}{Shortlist and Selection}}
\label{ch:shortlist}

Na het opstellen van de longlist in hoofdstuk~\ref{ch:longlist}, werden de geïdentificeerde cloudopslagoplossingen geëvalueerd aan de hand van de gedetailleerde en geprioriteerde vereisten uit de requirementsanalyse (Hoofdstuk 4). Het doel van deze fase was om de meest geschikte oplossing voor Aquarius Zwemclub Lebbeke (AZL) te selecteren, rekening houdend met de specifieke context en technische omgeving van de club.

\section*{Filteren van de Longlist} 

Op basis van de requirementsanalyse werden enkele oplossingen uit de longlist minder prioritair voor AZL:
\begin{itemize}
    \item \textbf{Dropbox:} Werd uitgesloten wegens de reeds vastgestelde fundamentele problemen rond toegangsbeheer, gebrek aan integratie, en de vereiste voor individuele accounts, die de kern vormen van de onderzoeksvraag.
    \item \textbf{Microsoft Azure \& Google Cloud Storage:} Hoewel krachtig, werden deze als minder ideaal beschouwd voor AZL vanwege de potentieel complexe prijsstructuren voor een kleine VZW en de mogelijke overhead in beheer vergeleken met meer gestroomlijnde oplossingen. AZL heeft ook geen bestaande infrastructuur op deze platformen.
\end{itemize}

Dit resulteerde in een shortlist van drie meer veelbelovende kandidaten:
\begin{itemize}
    \item Amazon S3
    \item DigitalOcean Spaces
    \item Nextcloud
\end{itemize}

\section*{Vergelijkende Analyse en Selectie}

De drie overgebleven kandidaten werden vervolgens vergeleken op basis van de meest kritische criteria voor AZL:

\begin{itemize}
    \item \textbf{Integratie met Bestaande Infrastructuur:} Zowel Amazon S3 als DigitalOcean Spaces bieden S3-compatibele API's, wat de technische integratie met de Angular/Node.js applicatie vereenvoudigt. Nextcloud biedt ook API's, maar vereist mogelijk meer specifieke implementatie. Het cruciale voordeel hier ligt bij \textbf{DigitalOcean Spaces}, aangezien AZL's webapplicatie en website reeds op DigitalOcean worden gehost. Dit minimaliseert netwerklatentie en potentiële data transfer kosten tussen applicatie en opslag, en centraliseert het beheer en de facturatie.
    \item \textbf{Kostenbeheersing:} Amazon S3 heeft een 'pay-as-you-go' model dat complex kan worden. Nextcloud (self-hosted) heeft geen licentiekosten, maar brengt infrastructuur- en onderhoudskosten met zich mee. \textbf{DigitalOcean Spaces} biedt een zeer voorspelbaar en eenvoudig prijsmodel (een vast bedrag per maand voor een ruime hoeveelheid opslag en dataverkeer), wat ideaal is voor het budgetbeheer van een VZW zoals AZL. Extra kosten voor API-requests, zoals bij S3, zijn er niet.
    \item \textbf{Beheer en Onderhoud:} Amazon S3 biedt zeer uitgebreide, maar potentieel complexe, beheermogelijkheden. Nextcloud vereist actieve updates en beheer, vooral bij self-hosting. \textbf{DigitalOcean Spaces} staat bekend om zijn eenvoudige beheerinterface, geïntegreerd in het bestaande DigitalOcean controlepaneel dat de webmaster al gebruikt. De S3-compatibiliteit maakt het ook mogelijk standaard tools te gebruiken.
    \item \textbf{Toegangsbeheer \& Gebruiksvriendelijkheid:} Alle drie de opties bieden mogelijkheden voor toegangsbeheer. S3 en Spaces doen dit via API keys, policies en pre-signed URLs, wat goed integreerbaar is in de backend. Nextcloud biedt een meer gebruikersgerichte interface voor beheer. Voor de eindgebruikers (lesgevers) zal de toegang echter voornamelijk via de aangepaste webapplicatie verlopen, waardoor de directe interface van de opslagdienst minder relevant is. De focus ligt op een backend die veilig en efficiënt rechten kan beheren, wat met de S3-compatibele API's van S3 en Spaces goed mogelijk is.
\end{itemize}

\textbf{Selectie:}
Op basis van deze vergelijking wordt \textbf{DigitalOcean Spaces} geselecteerd als de meest geschikte cloudopslagoplossing voor AZL. De doorslaggevende factoren zijn:
\begin{enumerate}
    \item De \textbf{synergie} met de bestaande hostinginfrastructuur bij DigitalOcean, wat leidt tot technische en operationele voordelen.
    \item Het \textbf{eenvoudige en voorspelbare prijsmodel}, wat cruciaal is voor een VZW.
    \item De \textbf{S3-compatibele API}, die een relatief eenvoudige technische integratie mogelijk maakt.
    \item Het \textbf{gebruiksgemak} van het beheerplatform voor de webmaster.
\end{enumerate}

\textbf{Backup Optie:}
Mocht tijdens de Proof of Concept (PoC) fase blijken dat DigitalOcean Spaces onverwachte beperkingen heeft voor AZL, wordt \textbf{Nextcloud} (eventueel in een gehoste variant om de beheerslast te beperken) beschouwd als de primaire back-up optie, vanwege zijn flexibiliteit en controle over data. Amazon S3 blijft een technisch valide, maar minder geprefereerde, optie vanwege de kostencomplexiteit en het ontbreken van directe synergie met de bestaande hosting.

De volgende stap is het ontwikkelen van een Proof of Concept met DigitalOcean Spaces.
