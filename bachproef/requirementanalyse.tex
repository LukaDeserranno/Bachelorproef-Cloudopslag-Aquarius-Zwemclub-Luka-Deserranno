\chapter{\IfLanguageName{dutch}{Requirement Analyse}{Requirement Analysis}}
\label{ch:requirement-analyse}

Dit hoofdstuk beschrijft het proces en de resultaten van de requirementsanalysefase, die cruciaal is voor het selecteren van een cloudopslagoplossing die effectief voldoet aan de behoeften van Aquarius Zwemclub Lebbeke (AZL). Deze fase omvatte nauwe samenwerking met de webmaster van AZL, die de technische systemen beheert, om een grondig begrip te verzekeren van de huidige uitdagingen en toekomstige noden.

Het primaire doel van deze analyse was het definiëren van de functionele en niet-functionele vereisten voor een nieuw cloudopslagsysteem ter vervanging van de huidige Dropbox-oplossing. De bestaande Dropbox-opstelling kent verschillende beperkingen, waaronder verouderde toegangsrechten, gebrek aan integratie met AZL's webapplicatie en WordPress-site, en toegankelijkheidsproblemen voor lesgevers zonder Dropbox-account, wat leidt tot verhoogde administratieve werklast.

Kerngebieden die tijdens deze fase werden onderzocht, omvatten:

\begin{itemize}
    \item \textbf{Toegangsbeheer:} Het bepalen hoe efficiënt en veilig toegang voor lesgevers kan worden beheerd, waarbij rechten dynamisch kunnen worden bijgewerkt op basis van hun status (bv. starten of stoppen bij de club). De oplossing moet gedetailleerde controle mogelijk maken zodat gebruikers enkel toegang hebben tot relevante informatie. Een kernvereiste is het mogelijk maken van toegang voor lesgevers zonder hen te dwingen extra accounts aan te maken.
    \item \textbf{Integratie:} Het evalueren van de mogelijkheden voor naadloze integratie met de bestaande IT-infrastructuur van AZL, specifiek de op maat gemaakte webapplicatie (Angular/Node.js) en de publieke WordPress-website.
    \item \textbf{Gebruiksvriendelijkheid:} Verzekeren dat de oplossing gebruiksvriendelijk is voor lesgevers, die mogelijk wisselende niveaus van technische expertise hebben, alsook beheersbaar is voor de webmaster.
    \item \textbf{Beveiliging en Privacy:} Implementeren van duidelijke machtigingsstructuren om gegevensprivacy te beschermen en te verzekeren dat lesgevers enkel toegang hebben tot geautoriseerde bestanden en mappen.
    \item \textbf{Kosteneffectiviteit:} Analyseren van de kosten verbonden aan potentiële oplossingen in relatie tot het budget van AZL, rekening houdend met het feit dat de huidige oplossing gratis is.
    \item \textbf{Onderhoudbaarheid:} Selecteren van een oplossing die op lange termijn eenvoudig te beheren en te onderhouden is door de webmaster.
\end{itemize}

Om deze vereisten te structureren, werd de MoSCoW-methode toegepast om ze te prioriteren in 'Must Have', 'Should Have', 'Could Have', en 'Won't Have' categorieën. Deze prioritering zorgt ervoor dat de geselecteerde oplossing primair de meest kritieke noden van AZL aanpakt.

Het resultaat van deze fase is een gedetailleerde en geprioriteerde lijst van vereisten, die dient als basis voor het evalueren van potentiële cloudopslagoplossingen in de volgende fasen (Longlist en Shortlist). Deze gestructureerde aanpak beoogt te garanderen dat de uiteindelijk gekozen oplossing perfect aansluit bij de operationele noden en technische omgeving van AZL.