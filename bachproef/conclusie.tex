%%=============================================================================
%% Conclusie
%%=============================================================================

\chapter{Conclusie}%
\label{ch:conclusie}

In deze bachelorproef werd onderzocht welke cloudopslagoplossing Aquarius Zwemclub Lebbeke (AZL) de beste balans biedt tussen gebruiksvriendelijkheid, kostenbeheersing, integratie met bestaande systemen en efficiënt gebruikersbeheer. Vanuit de centrale onderzoeksvraag werd in kaart gebracht welke vereisten AZL stelt aan een dergelijk systeem, en werd een selectieproces doorlopen op basis van functionele en niet-functionele criteria. De MoSCoW-methode werd toegepast om deze criteria te structureren en te prioriteren.

Mijn bijdrage aan het onderzoeksdomein ligt in het opstellen van een praktijkgericht selectiekader voor cloudopslag in kleinschalige organisaties met beperkte middelen. Daarnaast werd een proof of concept (PoC) ontwikkeld waarin DigitalOcean Spaces werd geïntegreerd in een bestaande Angular/Node.js-omgeving, voorzien van een fijnmazig en uitbreidbaar toegangsbeheersysteem. Deze architectuur bewijst dat moderne cloudopslagfunctionaliteit, inclusief beveiliging en granulariteit van toegang, ook zonder dure licenties of complexe infrastructuur realiseerbaar is. Daarmee biedt dit onderzoek concrete meerwaarde voor vrijwilligersverenigingen en sportclubs met vergelijkbare IT-behoeften.

De uitkomst bevestigt grotendeels de initiële verwachtingen: het technisch integreren van een alternatieve opslagoplossing bleek haalbaar en efficiënt. Toch zijn er nog onduidelijkheden en beperkingen. Zo werd de oplossing nog niet getest door eindgebruikers (lesgevers of beheerders) en bleef integratie met de publieke WordPress-site buiten scope. Daarnaast kunnen er bij productiegebruik nog uitdagingen opduiken rond gebruiksgemak, logging en foutafhandeling, die pas zichtbaar worden in een reële werkomgeving.

Het onderzoek roept ook nieuwe vragen op. Zo is het interessant om te onderzoeken hoe de oplossing zich gedraagt bij schaalvergroting of piekbelasting. Ook zou een uitbreiding naar federated sharing of mobiele toegang kunnen bijdragen tot bredere inzetbaarheid. Ten slotte is het relevant om te verkennen hoe een dergelijk systeem eenvoudig overdraagbaar gemaakt kan worden binnen vrijwilligersorganisaties met frequente wissels in beheer.

Samenvattend biedt deze bachelorproef een solide eerste stap naar een toekomstbestendige cloudstrategie voor AZL en vergelijkbare verenigingen, met ruimte voor verdere validatie, uitbreiding en optimalisatie in een vervolgtraject.

