%%=============================================================================
%% Conclusie
%%=============================================================================

\chapter{Conclusie}
\label{ch:conclusie}

In deze bachelorproef wordt onderzocht welke cloudopslagoplossing Aquarius Zwemclub Lebbeke (AZL) de beste balans biedt tussen gebruiksvriendelijkheid, kostenbeheersing, integratie met bestaande systemen en efficiënt gebruikersbeheer. Vanuit de centrale onderzoeksvraag worden de vereisten van AZL in kaart gebracht en wordt een selectieproces doorlopen op basis van functionele en niet-functionele criteria. De MoSCoW-methode structureert en prioriteert deze criteria.

De bijdrage van dit onderzoek ligt in het opstellen van een praktijkgericht selectiekader voor cloudopslag in kleinschalige organisaties met beperkte middelen. Daarnaast wordt een proof of concept (PoC) ontwikkeld waarin DigitalOcean Spaces geïntegreerd is in een bestaande Angular/Node.js-omgeving, voorzien van een fijnmazig en uitbreidbaar toegangsbeheersysteem. Deze architectuur toont aan dat moderne cloudopslagfunctionaliteit — inclusief beveiliging en fijnmazige toegangscontrole — ook zonder dure licenties of complexe infrastructuur realiseerbaar is. Daarmee biedt dit werk concrete meerwaarde voor vrijwilligersverenigingen en sportclubs met gelijkaardige IT-behoeften.

De resultaten bevestigen grotendeels de initiële verwachtingen: het technisch integreren van een alternatieve opslagoplossing blijkt haalbaar en efficiënt. De integratie met de publieke WordPress-site valt buiten scope, maar vormt een logische volgende stap. In productie kunnen nog aandachtspunten opduiken rond gebruiksgemak, logging en foutafhandeling, waarvoor het systeem ruimte laat om verder geoptimaliseerd te worden.

Het onderzoek werpt ook nieuwe vragen op. Zo is het relevant om te verkennen hoe het systeem zich gedraagt bij schaalvergroting of piekbelasting. Een uitbreiding naar \textbf{federated sharing} — waarbij bestanden gedeeld worden over verschillende organisaties zonder dat ze op één centrale locatie worden opgeslagen — kan eveneens bijdragen tot bredere inzetbaarheid. Verder is het waardevol om te onderzoeken hoe het systeem eenvoudig overdraagbaar kan blijven binnen vrijwilligersorganisaties, waar regelmatig wissels in beheer plaatsvinden.

Samenvattend biedt deze bachelorproef een solide basis voor een toekomstgerichte cloudstrategie binnen AZL en soortgelijke verenigingen, met voldoende ruimte voor verdere validatie, uitbreiding en optimalisatie.
