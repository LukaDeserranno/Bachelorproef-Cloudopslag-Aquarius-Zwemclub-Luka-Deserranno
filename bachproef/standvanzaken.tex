\chapter{\IfLanguageName{dutch}{Stand van zaken}{State of the art}}%
\label{ch:stand-van-zaken}

% Tip: Begin elk hoofdstuk met een paragraaf inleiding die beschrijft hoe
% dit hoofdstuk past binnen het geheel van de bachelorproef. Geef in het
% bijzonder aan wat de link is met het vorige en volgende hoofdstuk.

% Pas na deze inleidende paragraaf komt de eerste sectiehoofding.

Deze literatuurstudie onderzoekt beschikbare cloudopslagoplossingen die voldoen aan de specefieke behoeften van AZL. 
De focus ligt op integratiemogelijkheden met de bestaande Angular frontend en Node.js backend, gebruiksvriendelijkheid, efficiënt toegangsbeheer en kosteneffectiviteit. 
Deze studie vormt de basis voor het selecteren van een geschikte cloudopslagoplossing voor AZL.

\section{Huidige IT-infrastructuur bij AZL}
Aquarius Zwemclub Lebbeke (AZL) maakt momenteel gebruik van een eenvoudige maar doeltreffende IT-infrastructuur:
\begin{itemize}
    \item \textbf{Hosting}: De webapplicatie gebouwd met Angular en Node.js zijn gehost op DigitalOcean.
    \item \textbf{E-mailservices}: De SMTP-server wordt gehost bij AWS, dit omdat deze functionaliteit niet beschikbaar is bij DigitalOcean.
    \item \textbf{Cloudopslag}: Voor het delen van bestanden tussen lesgevers wordt Dropbox gebruikt.
    \item \textbf{Authenticatie en autorisatie}: De webapplicatie maakt gebruik van een zelfgebouwd authenticatiesysteem, er wordt dus geen gebruik gemaakt van externe diensten. Binnen de applicatie wordt reeds gebruik gemaakt van rollen.
\end{itemize}
Hoewel deze infrastructuur functioneel is, kent AZL verschillende uitdagingen: er is geen integratie tussen de cloudopslag en de 
webapplicatie, en geautomatiseerd toegangsbeheer ontbreekt. Een complete oplossing kan deze problemen aanpakken 
door integratie tussen de cloudopslag en de applicaties mogelijk te maken, wat zal zorgen voor een gebruiksvriendelijkere en efficiëntere werking.

\section{Integratie van cloudopslag in webapplicaties}
\subsection{Architecturele overwegingen}
Bij het integreren van cloudopslag in webapplicaties is de architectuur een cruciale factor. Integratie kan worden gerealiseerd via drie primaire architecturele benaderingen:
\begin{itemize}
  \item \textbf{API-gebaseerde integratie}: De webapplicatie communiceert via API's met de cloudopslag.
  \item \textbf{SDK-gebaseerde integratie}: Software Development Kits (SDK's) vereenvoudigen de communicatie tussen applicatie en cloudopslag.
  \item \textbf{Hybride integratie}: Combinatie van API's en SDK's voor optimale flexibiliteit.
\end{itemize}

\section{Vergelijkende analyse van cloudopslagoplossingen}
Om een geschikte cloudopslagoplossing voor AZL te vinden, zijn verschillende cloudoplossingen beoordeeld op basis van criteria die specifiek aansluiten bij de uitdagingen en behoeften van de organisatie:
\begin{itemize}
    \item Geschikt zijn voor kleine tot middelgrote organisaties: De oplossing moet betaalbaar en beheersbaar zijn voor een sportvereniging met beperkte middelen.
    \item Naadloos geïntegreerd kunnen worden: De oplossing moet aansluiten op de bestaande WordPress-website en Angular-webapplicatie.
    \item Flexibel zijn in toegangsbeheer: De oplossing moet een efficiënte en veilige manier bieden om toegangsrechten te beheren, waarbij lesgevers eenvoudig kunnen worden toegevoegd of verwijderd.
    \item Gebruiksvriendelijk en technisch haalbaar zijn: De oplossing moet eenvoudig te gebruiken zijn voor lesgevers met beperkte technische kennis en flexibel genoeg voor de webmaster.
    \item Onderhoudsvriendelijk zijn: De oplossing moet eenvoudig te beheren zijn door de webmaster na de implementatie.
\end{itemize}

\subsection{S3-Compatibele Object Storage}

S3-compatibele cloudopslag biedt verschillende technische voordelen voor Angular-webapplicaties op het vlak van prestaties, beveiliging, integratie en kostenefficiëntie. Zo tonen Jamal et al.~\cite{Jamal2021Performance} aan dat S3-opslag goede prestaties levert qua snelheid, lage netwerklast genereert, en kostenbesparend is ten opzichte van alternatieve systemen. Hierdoor kunnen Angular-applicaties efficiënt bestanden verwerken in real time.

Op het gebied van prestaties wijzen Liang en Kozat~\cite{Liang2014FAST} op aanzienlijke dalingen in latentiepercentielen (76\%, 80\% en 85\% bij 2~MB-bestanden) dankzij het gebruik van erasure coding gecombineerd met parallelle verbindingen. Dit resulteert in een merkbare verhoging van de efficiëntie.

Beveiligingsaspecten worden versterkt door systemen zoals FADE. Tang et al.~\cite{Tang2010FADE, Tang2012Secure} illustreren hoe gegarandeerde bestandsverwijdering en fijnmazige toegangscontrole kunnen worden geïmplementeerd met minimale impact op prestaties en kosten. Bovendien komen S3-oplossingen tegemoet aan compliancevereisten. Колпаков en Петренко~\cite{kolpakov2018data} beschrijven methoden voor integratie die voldoen aan normen zoals PCI DSS, HIPAA/HITECH en FEDRAMP.

Tot slot ondersteunen S3-oplossingen diverse integratiemethoden—zoals directe API-aanroepen, proxyservers en overlaymechanismen—waardoor ontwikkelaars de flexibiliteit hebben om opslagfunctionaliteit te integreren zonder de Angular-businesslogica te belasten.

\textbf{Samengevat zijn de voornaamste technische voordelen:}
\begin{itemize}
    \item Verbeterde prestaties met lagere latentie en gunstige kostenefficiëntie.
    \item Verhoogde beveiliging door gegarandeerde verwijdering, encryptie en toegangscontrole.
    \item Flexibele integratiemethoden die de complexiteit van S3 abstraheren van de Angular-logica.
\end{itemize}

\subsubsection{Amazon S3}
\subsubsection{DigitalOcean}




