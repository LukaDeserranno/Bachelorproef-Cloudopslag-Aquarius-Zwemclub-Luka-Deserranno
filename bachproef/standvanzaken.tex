\chapter{\IfLanguageName{dutch}{Stand van zaken}{State of the art}}%
\label{ch:stand-van-zaken}

% Tip: Begin elk hoofdstuk met een paragraaf inleiding die beschrijft hoe
% dit hoofdstuk past binnen het geheel van de bachelorproef. Geef in het
% bijzonder aan wat de link is met het vorige en volgende hoofdstuk.

% Pas na deze inleidende paragraaf komt de eerste sectiehoofding.

Deze literatuurstudie onderzoekt beschikbare cloudopslagoplossingen die voldoen aan de specefieke behoeften van AZL. 
De focus ligt op integratiemogelijkheden met de bestaande Angular frontend en Node.js backend, gebruiksvriendelijkheid, efficiënt toegangsbeheer en kosteneffectiviteit. 
Deze studie vormt de basis voor het selecteren van een geschikte cloudopslagoplossing voor AZL.

\section{Huidige IT-infrastructuur bij AZL}
Aquarius Zwemclub Lebbeke (AZL) maakt momenteel gebruik van een eenvoudige maar doeltreffende IT-infrastructuur:
\begin{itemize}
    \item \textbf{Hosting}: De webapplicatie gebouwd met Angular en Node.js zijn gehost op DigitalOcean.
    \item \textbf{E-mailservices}: De SMTP-server wordt gehost bij AWS, dit omdat deze functionaliteit niet beschikbaar is bij DigitalOcean.
    \item \textbf{Cloudopslag}: Voor het delen van bestanden tussen lesgevers wordt Dropbox gebruikt.
    \item \textbf{Authenticatie en autorisatie}: De webapplicatie maakt gebruik van een zelfgebouwd authenticatiesysteem, er wordt dus geen gebruik gemaakt van externe diensten. Binnen de applicatie wordt reeds gebruik gemaakt van rollen.
\end{itemize}
Hoewel deze infrastructuur functioneel is, kent AZL verschillende uitdagingen: er is geen integratie tussen de cloudopslag en de 
webapplicatie, en geautomatiseerd toegangsbeheer ontbreekt. Een complete oplossing kan deze problemen aanpakken 
door integratie tussen de cloudopslag en de applicaties mogelijk te maken, wat zal zorgen voor een gebruiksvriendelijkere en efficiëntere werking.

\section{Integratie van cloudopslag in webapplicaties}
\subsection{Architecturele overwegingen}
Bij het integreren van cloudopslag in webapplicaties is de architectuur een cruciale factor. Integratie kan worden gerealiseerd via drie primaire architecturele benaderingen:
\begin{itemize}
  \item \textbf{API-gebaseerde integratie}: De webapplicatie communiceert via API's met de cloudopslag.
  \item \textbf{SDK-gebaseerde integratie}: Software Development Kits (SDK's) vereenvoudigen de communicatie tussen applicatie en cloudopslag.
  \item \textbf{Hybride integratie}: Combinatie van API's en SDK's voor optimale flexibiliteit.
\end{itemize}

\section{Vergelijkende analyse van cloudopslagoplossingen}
Om een geschikte cloudopslagoplossing voor AZL te vinden, zijn verschillende cloudoplossingen beoordeeld op basis van criteria die specifiek aansluiten bij de uitdagingen en behoeften van de organisatie:
\begin{itemize}
    \item Geschikt zijn voor kleine tot middelgrote organisaties: De oplossing moet betaalbaar en beheersbaar zijn voor een sportvereniging met beperkte middelen.
    \item Naadloos geïntegreerd kunnen worden: De oplossing moet aansluiten op de bestaande WordPress-website en Angular-webapplicatie.
    \item Flexibel zijn in toegangsbeheer: De oplossing moet een efficiënte en veilige manier bieden om toegangsrechten te beheren, waarbij lesgevers eenvoudig kunnen worden toegevoegd of verwijderd.
    \item Gebruiksvriendelijk en technisch haalbaar zijn: De oplossing moet eenvoudig te gebruiken zijn voor lesgevers met beperkte technische kennis en flexibel genoeg voor de webmaster.
    \item Onderhoudsvriendelijk zijn: De oplossing moet eenvoudig te beheren zijn door de webmaster na de implementatie.
\end{itemize}

\subsection{S3-Compatibele Object Storage}
Amazon S3 (Simple Storage Service), geïntroduceerd door AWS, was een van de eerste en meest invloedrijke object storage diensten op de markt \cite{S3CompatAngularIntro}. % Vervang S3CompatAngularIntro door je BibTeX key
De bijbehorende Application Programming Interface (API) is door zijn robuustheid, schaalbaarheid en vroege adoptie uitgegroeid tot een \textit{de facto} standaard binnen de cloud storage industrie \cite{S3CompatAngularStandard}. % Vervang S3CompatAngularStandard door je BibTeX key
Dit heeft aanzienlijke voordelen voor ontwikkelaars en organisaties:

\begin{itemize}
    \item \textbf{Ecosysteem \& Tooling:} Er is een breed ecosysteem ontstaan rondom de S3 API \cite{S3CompatAngularEcosystem}. % Vervang S3CompatAngularEcosystem door je BibTeX key
    Ontwikkelaars kunnen gebruikmaken van uitgebreide Software Development Kits (SDK's) voor diverse programmeertalen (zoals JavaScript voor Angular/Node.js) \cite{S3CompatAngularSDKs}, % Vervang S3CompatAngularSDKs door je BibTeX key
    talloze libraries, en command-line tools die S3 ondersteunen.
    \item \textbf{Kennis \& Overdraagbaarheid:} Investering in kennis over het werken met de S3 API is duurzaam, omdat deze kennis vaak direct toepasbaar is bij andere providers die de S3 API ondersteunen \cite{S3CompatAngularKnowledgeTransfer}. % Vervang S3CompatAngularKnowledgeTransfer door je BibTeX key
    \item \textbf{Concurrentie \& Keuzevrijheid:} Diverse andere cloud providers bieden object storage diensten aan die compatibel zijn met de S3 API \cite{S3CompatAngularOtherProviders}. % Vervang S3CompatAngularOtherProviders door je BibTeX key
    Deze S3-compatibele oplossingen, zoals DigitalOcean Spaces \cite{BPVoorstelDOSpaces} % Vervang BPVoorstelDOSpaces door je BibTeX key
    dat ook in dit onderzoek wordt bekeken, concurreren vaak op basis van prijs (bijvoorbeeld eenvoudigere kostenmodellen \cite{S3CompatAngularPricing} % Vervang S3CompatAngularPricing door je BibTeX key
    of lagere kosten voor dataverkeer), gebruiksgemak, of specifieke features, terwijl ze toch de voordelen van de gestandaardiseerde S3 API bieden.
\end{itemize}

Voor AZL betekent dit dat kiezen voor een oplossing die de S3 API gebruikt (ofwel AWS S3 zelf, of een compatibele aanbieder) een goede basis legt voor integratie met de bestaande Angular/Node.js applicatie, dankzij de beschikbare tools en SDK's. Hieronder analyseren we eerst AWS S3 en vervolgens een representatief S3-compatibel alternatief (DigitalOcean Spaces) aan de hand van de gestelde criteria.

\subsubsection{Amazon S3}
\subsubsection{DigitalOcean}




