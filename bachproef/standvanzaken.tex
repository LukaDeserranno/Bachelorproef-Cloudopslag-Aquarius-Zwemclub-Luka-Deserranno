\chapter{\IfLanguageName{dutch}{Stand van zaken}{State of the art}}%
\label{ch:stand-van-zaken}

% Tip: Begin elk hoofdstuk met een paragraaf inleiding die beschrijft hoe
% dit hoofdstuk past binnen het geheel van de bachelorproef. Geef in het
% bijzonder aan wat de link is met het vorige en volgende hoofdstuk.

% Pas na deze inleidende paragraaf komt de eerste sectiehoofding.

Deze literatuurstudie onderzoekt beschikbare cloudopslagoplossingen die voldoen aan de specefieke behoeften van AZL. 
De focus ligt op integratiemogelijkheden met de bestaande Angular frontend en Node.js backend, gebruiksvriendelijkheid, efficiënt toegangsbeheer en kosteneffectiviteit. 
Deze studie vormt de basis voor het selecteren van een geschikte cloudopslagoplossing voor AZL.

\section{Huidige IT-infrastructuur bij AZL}
Aquarius Zwemclub Lebbeke (AZL) maakt momenteel gebruik van een eenvoudige maar doeltreffende IT-infrastructuur:
\begin{itemize}
    \item \textbf{Hosting}: De webapplicatie gebouwd met Angular en Node.js zijn gehost op DigitalOcean.
    \item \textbf{E-mailservices}: De SMTP-server wordt gehost bij AWS, dit omdat deze functionaliteit niet beschikbaar is bij DigitalOcean.
    \item \textbf{Cloudopslag}: Voor het delen van bestanden tussen lesgevers wordt Dropbox gebruikt.
    \item \textbf{Authenticatie en autorisatie}: De webapplicatie maakt gebruik van een zelfgebouwd authenticatiesysteem, er wordt dus geen gebruik gemaakt van externe diensten. Binnen de applicatie wordt reeds gebruik gemaakt van rollen.
\end{itemize}
Hoewel deze infrastructuur functioneel is, kent AZL verschillende uitdagingen: er is geen integratie tussen de cloudopslag en de 
webapplicatie, en geautomatiseerd toegangsbeheer ontbreekt. Een complete oplossing kan deze problemen aanpakken 
door integratie tussen de cloudopslag en de applicaties mogelijk te maken, wat zal zorgen voor een gebruiksvriendelijkere en efficiëntere werking.

\section{Integratie van cloudopslag in webapplicaties}
\subsection{Architecturele overwegingen}
Bij het integreren van cloudopslag in webapplicaties is de architectuur een cruciale factor. Integratie kan worden gerealiseerd via drie primaire architecturele benaderingen:
\begin{itemize}
  \item \textbf{API-gebaseerde integratie}: De webapplicatie communiceert via API's met de cloudopslag.
  \item \textbf{SDK-gebaseerde integratie}: Software Development Kits (SDK's) vereenvoudigen de communicatie tussen applicatie en cloudopslag.
  \item \textbf{Hybride integratie}: Combinatie van API's en SDK's voor optimale flexibiliteit.
\end{itemize}

\subsection{Frontend frameworks voor bestandsbeheer}
Voor de interactie en visualisatie van de cloudopslag in Angular zijn verschillende UI-frameworks beschikbaar. Dit maakt het ontwikkelen minder complex daar er al een component voor bestaat.

\subsubsection{ngx-explorer}
\texttt{ngx-explorer} is een Angular-component die uitgebreide ondersteuning biedt voor het beheren en visualiseren van bestanden in een cloudomgeving. De implementatie verloopt in enkele eenvoudige stappen:
  
\textbf{Configuratie en gebruik}: Plaats de component in je template met de tag \texttt{<ngx-explorer>}. Via inputparameters en events kun je onder andere:
  \begin{itemize}
    \item De startmap en toegangsrechten instellen.
    \item Thema’s en stijlen aanpassen middels CSS-overschrijvingen.
    \item Evenementen zoals bestandsselectie of uploadstatus monitoren.
  \end{itemize}

Wat betreft aanpasbaarheid biedt \texttt{ngx-explorer} een solide basis:
\begin{itemize}
  \item De standaard UI heeft een gemiddelde flexibiliteit, maar door gebruik te maken van Angular's component-architectuur en uitgebreide CSS-opties is het relatief eenvoudig om de look en feel aan te passen.
  \item De uitgebreide documentatie en actieve onderhoudsondersteuning zorgen ervoor dat ontwikkelaars snel problemen kunnen oplossen en de component kunnen uitbreiden naar specifieke behoeften.
  \item De component ondersteunt integratie met diverse cloudopslagdiensten via zowel API's als SDK's, wat een hybride integratie mogelijk maakt en de schaalbaarheid van de applicatie waarborgt.
\end{itemize}

\section{Vergelijkende analyse van cloudopslagoplossingen}
Om een geschikte cloudopslagoplossing voor AZL te vinden, zijn verschillende cloudoplossingen beoordeeld op basis van criteria die specifiek aansluiten bij de uitdagingen en behoeften van de organisatie:
\begin{itemize}
    \item Geschikt zijn voor kleine tot middelgrote organisaties: De oplossing moet betaalbaar en beheersbaar zijn voor een sportvereniging met beperkte middelen.
    \item Naadloos geïntegreerd kunnen worden: De oplossing moet aansluiten op de bestaande WordPress-website en Angular-webapplicatie.
    \item Flexibel zijn in toegangsbeheer: De oplossing moet een efficiënte en veilige manier bieden om toegangsrechten te beheren, waarbij lesgevers eenvoudig kunnen worden toegevoegd of verwijderd.
    \item Gebruiksvriendelijk en technisch haalbaar zijn: De oplossing moet eenvoudig te gebruiken zijn voor lesgevers met beperkte technische kennis en flexibel genoeg voor de webmaster.
    \item Onderhoudsvriendelijk zijn: De oplossing moet eenvoudig te beheren zijn door de webmaster na de implementatie.
\end{itemize}