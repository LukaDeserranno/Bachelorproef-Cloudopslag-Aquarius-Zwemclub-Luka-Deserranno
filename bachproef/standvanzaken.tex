\chapter{\IfLanguageName{dutch}{Stand van zaken}{State of the art}}%
\label{ch:stand-van-zaken}

% Tip: Begin elk hoofdstuk met een paragraaf inleiding die beschrijft hoe
% dit hoofdstuk past binnen het geheel van de bachelorproef. Geef in het
% bijzonder aan wat de link is met het vorige en volgende hoofdstuk.

% Pas na deze inleidende paragraaf komt de eerste sectiehoofding.

Deze literatuurstudie onderzoekt beschikbare cloudopslagoplossingen die voldoen aan de specefieke behoeften van AZL. 
De focus ligt op integratiemogelijkheden met de bestaande Angular frontend en Node.js backend, gebruiksvriendelijkheid, efficiënt toegangsbeheer en kosteneffectiviteit. 
Deze studie vormt de basis voor het selecteren van een geschikte cloudopslagoplossing voor AZL.

\section{Huidige IT-infrastructuur bij AZL}
Aquarius Zwemclub Lebbeke (AZL) maakt momenteel gebruik van een eenvoudige maar doeltreffende IT-infrastructuur:
\begin{itemize}
    \item \textbf{Hosting webapplicatie}: De interne webapplicatie, gebouwd met Angular en Node.js, is gehost op DigitalOcean.
    \item \textbf{Website en databanken}: De publieke WordPress-website, de bijhorende MySQL-databank en de e-mailadressen met het domein \texttt{@aquariuslebbeke.be} worden gehost bij One.com.
    \item \textbf{E-mailservices}: De SMTP-server voor transactionele e-mails van de webapplicatie wordt gehost bij AWS, aangezien DigitalOcean deze functionaliteit niet aanbiedt.
    \item \textbf{Cloudopslag}: Voor het delen van bestanden tussen lesgevers wordt momenteel gebruikgemaakt van Dropbox.
    \item \textbf{Authenticatie en autorisatie}: De webapplicatie maakt gebruik van een zelfgebouwd authenticatiesysteem zonder externe authenticatiediensten. Binnen de applicatie wordt reeds gebruikgemaakt van rollen.
\end{itemize}
Hoewel deze infrastructuur functioneel is, kent AZL verschillende uitdagingen: er is geen integratie tussen de cloudopslag en de webapplicatie, en geautomatiseerd toegangsbeheer ontbreekt. Een nieuwe oplossing kan deze problemen aanpakken door integratie tussen cloudopslag en applicaties mogelijk te maken, wat zal zorgen voor een gebruiksvriendelijkere en efficiëntere werking.

\section{Integratie van cloudopslag in webapplicaties}
\subsection{Architecturele overwegingen}
Bij het integreren van cloudopslag in webapplicaties is de architectuur een cruciale factor. Integratie kan worden gerealiseerd via drie primaire architecturele benaderingen:
\begin{itemize}
  \item \textbf{API-gebaseerde integratie}: De webapplicatie communiceert rechtstreeks via API's (Application Programming Interfaces) met de cloudopslag. Dit vereist handmatige aanroepen van API-endpoints en het verwerken van de responses.
  \item \textbf{SDK-gebaseerde integratie}: Software Development Kits (SDK's) bouwen verder op de onderliggende API's en bieden vooraf gemaakte functies en klassen. Ze vereenvoudigen de communicatie door complexe API-aanroepen te abstraheren, waardoor ontwikkelaars sneller en veiliger kunnen integreren zonder zelf de volledige logica te hoeven implementeren.
  \item \textbf{Hybride integratie}: Een combinatie van API's en SDK's voor optimale flexibiliteit. Bijvoorbeeld, standaardhandelingen gebeuren via de SDK, terwijl specifieke, geavanceerde functionaliteit direct via de API wordt aangesproken.
\end{itemize}

\section{Vergelijkende analyse van cloudopslagoplossingen}
Om een geschikte cloudopslagoplossing voor AZL te vinden, zijn verschillende cloudoplossingen beoordeeld op basis van criteria die specifiek aansluiten bij de uitdagingen en behoeften van de organisatie:
\begin{itemize}
    \item Geschikt zijn voor kleine tot middelgrote organisaties: De oplossing moet betaalbaar en beheersbaar zijn voor een sportvereniging met beperkte middelen.
    \item Naadloos geïntegreerd kunnen worden: De oplossing moet aansluiten op de bestaande WordPress-website en Angular-webapplicatie.
    \item Flexibel zijn in toegangsbeheer: De oplossing moet een efficiënte en veilige manier bieden om toegangsrechten te beheren, waarbij lesgevers eenvoudig kunnen worden toegevoegd of verwijderd.
    \item Gebruiksvriendelijk en technisch haalbaar zijn: De oplossing moet eenvoudig te gebruiken zijn voor lesgevers met beperkte technische kennis en flexibel genoeg voor de webmaster.
    \item Onderhoudsvriendelijk zijn: De oplossing moet eenvoudig te beheren zijn door de webmaster na de implementatie.
\end{itemize}

\subsection{Dropbox}
Hoewel Dropbox een gebruiksvriendelijke oplossing biedt voor het delen van bestanden, voldoet het niet aan alle behoeften van AZL. 
Het gebrek aan integratie met de bestaande systemen zorgt voor extra werk voor de beheerders, en het ontbreken van geautomatiseerd 
toegangsbeheer maakt het moeilijk om de juiste rechten op elk tijdstip te garanderen voor alle gebruikers. Bovendien vereist Dropbox dat 
elke gebruiker een account aanmaakt, wat een extra drempel kan vormen voor lesgevers die niet vertrouwd zijn met de dienst.

Dropbox biedt via de Dropbox API de mogelijkheid om cloudopslagfunctionaliteiten, zoals bestandenbeheer en mappenbeheer, in webapplicaties te integreren \autocite{dropbox_api}. 
De API stelt ontwikkelaars in staat om bestanden te uploaden, downloaden, mappen te creëren en bestanden te delen, wat ideaal lijkt voor het delen van 
bestanden tussen lesgevers bij Aquarius Zwemclub Lebbeke (AZL). Echter, de meeste van deze functies vereisen het gebruik van OAuth 2.0 voor de authenticatie 
en autorisatie van gebruikers \autocite{dropbox_oauth}.

De webapplicatie voor de lesgevers bij AZL maakt geen gebruik van OAuth 2.0 voor authenticatie. Dit betekent dat, in de huidige opzet van de applicatie, 
het technisch niet eenvoudig is om gebruikers veilig toegang te geven tot hun Dropbox-accounts zonder dat hun inloggegevens gedeeld worden. Om Dropbox te 
integreren, zou het volledige authenticatiesysteem van de applicatie moeten worden omgebouwd om OAuth 2.0-ondersteuning toe te voegen \autocite{dropbox_oauth}. Dit houdt in dat de 
applicatie de benodigde infrastructuur moet opzetten om gebruikers via OAuth 2.0 te authenticeren, wat een significante verandering in het huidige systeem vereist.

Zonder OAuth 2.0 zou de applicatie afhankelijk moeten zijn van andere methoden om verbinding te maken met Dropbox, zoals het handmatig verstrekken van API-sleutels 
of toegangstokens. Deze benadering is echter niet aanbevolen, omdat het in strijd is met moderne beveiligingspraktijken. Het ontbreken van OAuth 2.0 betekent 
dat de applicatie niet de benodigde bescherming kan bieden voor gebruikersgegevens, waardoor de vertrouwelijkheid en de integriteit van opgeslagen bestanden 
in gevaar komen.

Daarnaast is de opslagcapaciteit van een gratis persoonlijk Dropbox-account beperkt tot 2 GB, wat mogelijk onvoldoende is voor de toekomstige behoeften van AZL, 
vooral wanneer meerdere gebruikers bestanden moeten delen en opslaan. Het beheer van gebruikersmappen en het toekennen van individuele toegang tot deze 
mappen is ook problematisch, omdat de API geen methode biedt om toegang per gebruiker te regelen zonder OAuth 2.0.

Gezien de beperkingen van een gratis persoonlijk Dropbox-account, het ontbreken van OAuth 2.0-authenticatie, en de noodzaak om het gehele authenticatiesysteem 
om te bouwen, is de integratie van Dropbox in de webapplicatie voor lesgevers technisch haalbaar, maar het vereist aanzienlijke aanpassingen aan de infrastructuur 
en beveiligingsstructuur van de applicatie.

\subsection{S3-Compatibele Object Storage}

S3-compatibele cloudopslag biedt verschillende technische voordelen voor Angular-webapplicaties op het vlak van prestaties, beveiliging, integratie en kostenefficiëntie. Zo tonen Jamal et al.~\cite{Jamal2021Performance} aan dat S3-opslag goede prestaties levert qua snelheid, lage netwerklast genereert, en kostenbesparend is ten opzichte van alternatieve systemen. Hierdoor kunnen Angular-applicaties efficiënt bestanden verwerken in real time.

Op het gebied van prestaties wijzen Liang en Kozat~\cite{Liang2014FAST} op aanzienlijke dalingen in latentiepercentielen (76\%, 80\% en 85\% bij 2~MB-bestanden) dankzij het gebruik van erasure coding gecombineerd met parallelle verbindingen. Dit resulteert in een merkbare verhoging van de efficiëntie.

Beveiligingsaspecten worden versterkt door systemen zoals FADE. Tang et al.~\cite{Tang2010FADE, Tang2012Secure} illustreren hoe gegarandeerde bestandsverwijdering en fijnmazige toegangscontrole kunnen worden geïmplementeerd met minimale impact op prestaties en kosten. Bovendien komen S3-oplossingen tegemoet aan compliancevereisten. Колпаков en Петренко~\cite{kolpakov2018data} beschrijven methoden voor integratie die voldoen aan normen zoals PCI DSS, HIPAA/HITECH en FEDRAMP.

Tot slot ondersteunen S3-oplossingen diverse integratiemethoden—zoals directe API-aanroepen, proxyservers en overlaymechanismen—waardoor ontwikkelaars de flexibiliteit hebben om opslagfunctionaliteit te integreren zonder de Angular-businesslogica te belasten.

\textbf{Samengevat zijn de voornaamste technische voordelen:}
\begin{itemize}
    \item Verbeterde prestaties met lagere latentie en een gunstige kostenefficiëntie door gebruik van S3-compatibele opslag.
    \item Verhoogde beveiliging door gegarandeerde verwijdering, encryptie en fijnmazige toegangscontrole op bucketniveau.
    \item Backend-gebaseerde integratie: de Angular frontend communiceert uitsluitend met de eigen backend-API, die de interactie met de cloudopslag verzorgt. Hierdoor blijft de opslaginfrastructuur afgeschermd van de clientzijde, wat de beveiliging en beheersbaarheid aanzienlijk verhoogt.
\end{itemize}

\subsubsection{Amazon S3}
Amazon Simple Storage Service (S3) is een schaalbare objectopslagdienst van Amazon Web Services (AWS) die geschikt is voor toepassingen met uiteenlopende opslagbehoeften. Voor AZL vormt S3 een mogelijk alternatief voor Dropbox, mede dankzij zijn flexibiliteit, veiligheid en integratiemogelijkheden met webapplicaties zoals de bestaande Angular frontend en Node.js backend.

S3 slaat gegevens op in \textit{buckets}, waarin objecten – bestaande uit data en metadata – uniek geïdentificeerd worden via sleutels. Deze structuur laat toe om een logische mappenhiërarchie te simuleren door gebruik van voorvoegsels, zonder gebonden te zijn aan de beperkingen van traditionele bestandssystemen~\cite{aws_docs_structuur}. Daarnaast biedt S3 via een RESTful API directe integratie met webapplicaties, ondersteund door SDK's en tools zoals AWS Amplify~\cite{aws_docs_amplify}.

Voor AZL is vooral de ondersteuning voor \textit{pre-signed URLs} relevant. Deze maken veilige frontend uploads mogelijk zonder dat AWS-credentials in de clientcode moeten worden opgenomen~\cite{aws_presigned}. De backend (Node.js) kan instaan voor het genereren van deze tijdelijke links, waarmee de Angular frontend rechtstreeks bestanden kan opslaan of ophalen.

Qua schaalbaarheid en betrouwbaarheid biedt Amazon S3 sterke garanties: een beschikbaarheid van 99,99\% en een duurzaamheid van 99,999999999\%~\cite{aws_availability}. Bovendien kunnen verschillende opslagklassen worden gebruikt afhankelijk van de toegangspatronen, gaande van S3 Standard voor frequente toegang tot Glacier Deep Archive voor langetermijnarchivering~\cite{aws_classes}.

Toegangsbeheer wordt in S3 geregeld via \textit{IAM policies}, \textit{bucket policies} en optioneel \textit{access points}. Deze mechanismen maken fijnmazige controle mogelijk over wie welke bestanden mag lezen, schrijven of verwijderen, wat toelaat om rolgebaseerde toegang toe te passen voor lesgevers en beheerders~\cite{aws_accesscontrol}.

Tenslotte is S3 kosteneffectief voor kleine organisaties zoals AZL dankzij het \textit{pay-as-you-go} prijsmodel. Er wordt enkel betaald voor wat effectief wordt opgeslagen en verbruikt, met bijkomende mogelijkheden om via lifecycle policies automatisch minder gebruikte data naar goedkopere opslagklassen te verplaatsen~\cite{aws_costs}.

Amazon S3 sluit hierdoor nauw aan bij de eisen van AZL: veilige en flexibele opslag, eenvoudige integratie in de bestaande infrastructuur, gedetailleerd toegangsbeheer en een schaalbaar kostenmodel.

\subsubsection{DigitalOcean Spaces}
DigitalOcean Spaces is een objectopslagdienst die zich richt op eenvoud, schaalbaarheid en betaalbaarheid. De dienst is S3-compatibel en biedt daarmee een vertrouwde interface voor ontwikkelaars die reeds ervaring hebben met Amazon Web Services. Spaces ondersteunt het opslaan van ongestructureerde data zoals afbeeldingen, video's, back-ups en statische webbestanden, wat het uitermate geschikt maakt voor organisaties zoals AZL die veelvuldig met media- en documentbestanden werken~\cite{do_product}.

Een belangrijk voordeel van DigitalOcean Spaces is het voorspelbare prijsmodel. Voor \$5 per maand krijgt men 250~GiB aan opslag en 1~TiB aan uitgaand dataverkeer~\cite{do_pricing}. In tegenstelling tot Amazon S3 worden er geen extra kosten aangerekend voor API-verzoeken, wat voordelig is voor webapplicaties met veel interactie tussen frontend en opslag. Ook is er een ingebouwd Content Delivery Network (CDN) inbegrepen, dat statische content versneld kan leveren aan gebruikers wereldwijd~\cite{do_features}.

De integratie met een Angular frontend en Node.js backend is technisch haalbaar dankzij de S3-compatibele API. Via de AWS SDK voor JavaScript kunnen CRUD-operaties worden uitgevoerd op objecten in Spaces~\cite{do_sdk}. Voor AZL kan dit betekenen dat uploads en downloads veilig verlopen via pre-signed URLs, gegenereerd door de backend, zonder dat gevoelige sleutels op de client worden blootgesteld~\cite{do_security}. De frontend kan daarnaast direct publieke assets serveren, bijvoorbeeld afbeeldingen of documenten, met behulp van de CDN-url's.

Een bijkomend voordeel is dat AZL hun webapplicatie reeds host op DigitalOcean, waardoor dataoverdracht binnen het ecosysteem snel en zonder extra kosten kan verlopen~\cite{do_latency}. Dit verlaagt de netwerklatentie en vereenvoudigt facturatie, omdat infrastructuur- en opslagkosten gecentraliseerd zijn.

Spaces biedt toegangsbeheer via per-bucket toegangssleutels en bucket policies. Elk object kan publiek of privé worden ingesteld. Daarnaast ondersteunt DigitalOcean de AWS Signature v4 authenticatiemethode, waarmee veilige API-verzoeken kunnen worden gedaan~\cite{do_accesskeys}. Encryptie wordt zowel tijdens transport als bij opslag toegepast, conform industriestandaarden~\cite{do_security_model}.

Tot slot is Spaces gebruiksvriendelijk voor zowel technische als niet-technische gebruikers. Het DigitalOcean Controlepaneel biedt drag-and-drop functionaliteit voor bestandbeheer, terwijl geavanceerde taken mogelijk zijn via CLI-tools zoals s3cmd of de AWS CLI~\cite{do_ui}. Dit maakt het beheer toegankelijk voor lesgevers en vrijwilligers van AZL die minder technische ervaring hebben.

Gezien de eenvoudige integratie, het transparante kostenmodel en de S3-compatibiliteit vormt DigitalOcean Spaces een sterke kandidaat als cloudopslagoplossing voor AZL. Het is met name aantrekkelijk voor kleine tot middelgrote organisaties die streven naar een evenwicht tussen functionaliteit, veiligheid en budgetbeheer.


\subsection{Nextcloud}
Nextcloud is een open-source cloudopslagoplossing die bekend staat om zijn flexibiliteit en controle over gegevens. Het biedt een privé cloudopslagplatform dat bedrijven in staat stelt om hun eigen infrastructuur te gebruiken voor het hosten van hun gegevens, in plaats van afhankelijk te zijn van externe providers. Nextcloud ondersteunt integraties met externe diensten zoals Dropbox, Google Drive en Amazon S3, en biedt uitgebreide mogelijkheden voor bestandssynchronisatie en delen \autocite{nextcloud_features}.

Voor AZL zou Nextcloud de mogelijkheid bieden om de volledige controle over hun gegevens te behouden, aangezien het lokaal gehost kan worden of via een gehoste oplossing kan worden gebruikt. Het biedt ook een uitgebreid systeem voor toegangsbeheer, inclusief ondersteuning voor single sign-on (SSO) en federated cloud, waarmee bestanden veilig tussen verschillende instellingen kunnen worden gedeeld \autocite{nextcloud_sso}. Daarnaast biedt Nextcloud een API waarmee de cloudopslag eenvoudig kan worden geïntegreerd met andere applicaties, wat de interoperabiliteit met de bestaande systemen van AZL vergemakkelijkt.

Een ander voordeel van Nextcloud is de mogelijkheid om de kosten te beheersen, aangezien het open-source is en zonder licentiekosten kan worden geïmplementeerd. Er kunnen echter kosten ontstaan bij het hosten van de serverinfrastructuur en het onderhouden van de software. Ook moet de juiste technische expertise aanwezig zijn om de oplossing op te zetten en te beheren.

\subsection{Microsoft Azure}
Microsoft Azure biedt een breed scala aan cloudopslagoplossingen, waaronder Blob Storage voor objectopslag, File Storage voor bestandsgebaseerde opslag, en Disk Storage voor virtuele machines. Azure Blob Storage biedt een schaalbare en betrouwbare oplossing voor het opslaan van grote hoeveelheden ongestructureerde data, wat het geschikt maakt voor het beheren van bestanden zoals documenten, afbeeldingen en video's \autocite{azure_blob}.

Azure biedt ook uitgebreide integratiemogelijkheden, waaronder een sterk toegangsbeheer met Azure Active Directory (AD), wat het voor AZL mogelijk maakt om naadloos toegang te verlenen tot opslag op basis van gebruikersrollen en organisatorische vereisten \autocite{azure_ad}. Daarnaast maakt Azure gebruik van wereldwijde datacenters, waardoor het platform hoge beschikbaarheid en redundantie biedt.

Een potentieel nadeel van Azure is de prijsstelling, die op basis van gebruik kan variëren, wat moeilijk voorspelbaar kan zijn voor kleinere organisaties zoals AZL. De complexiteit van de dienst kan ook een uitdaging zijn voor organisaties die geen uitgebreide IT-afdeling hebben \autocite{azure_pricing}.

\subsection{Google Cloud Storage}
Google Cloud Storage biedt schaalbare en betrouwbare opslag voor een breed scala aan gegevens, van ongestructureerde data tot back-ups en archivering. Het biedt een eenvoudig te gebruiken API voor het uploaden, downloaden en beheren van bestanden \autocite{google_storage}. Google Cloud Storage heeft ook geavanceerde beveiligingsfuncties, zoals encryptie van gegevens in rust en tijdens overdracht, en uitgebreide toegangscontrole via Google Cloud Identity and Access Management (IAM) \autocite{google_iam}.

Een belangrijk voordeel van Google Cloud Storage is de sterke integratie met andere Google-diensten zoals Google Drive, Firebase en BigQuery, wat de interoperabiliteit met andere systemen vergemakkelijkt. Voor AZL zou Google Cloud Storage een gebruiksvriendelijke oplossing kunnen bieden, met robuuste beveiligingsmaatregelen en een sterke focus op prestaties.

Net als bij andere grote cloudproviders, is de prijsstructuur van Google Cloud Storage complex, en de kosten kunnen oplopen naarmate de opslagbehoeften van AZL groeien \autocite{google_pricing}. De schaalvoordelen van Google kunnen echter voordelig zijn voor grotere organisaties, terwijl kleinere organisaties mogelijk baat hebben bij alternatieven zoals Nextcloud of DigitalOcean Spaces.


\section{UI-componenten voor cloudintegratie in Angular}

De integratie van cloudopslagdiensten zoals Amazon S3, DigitalOcean Spaces, Nextcloud, Microsoft Azure en Google Cloud Storage in Angular-webapplicaties vereist zorgvuldige overweging van de benodigde UI-componenten. Deze componenten moeten directe interactie met de clouddiensten mogelijk maken zonder uitgebreide backend-aanpassingen. In deze sectie analyseren we de beschikbare UI-componenten, packages en implementatiemethoden die deze directe integratie mogelijk maken.

\subsection{Clientzijdige integratiemethoden}

Voor directe integratie van cloudopslagdiensten zonder afhankelijkheid van backend-aanpassingen bestaan er meerdere benaderingen.

\subsubsection{Directe API-integratie met STS/Presigned URLs}

Een effectieve methode voor directe cloudopslagintegratie is het gebruik van beveiligde tijdelijke credentials of vooraf ondertekende URLs. Deze aanpak minimaliseert de backend-vereisten.

\textbf{Benodigde UI-componenten:}
\begin{itemize}
    \item Upload- en downloadcomponenten die met presigned URLs kunnen werken
    \item Bestandsselectie- en voortgangsweergave
    \item Bestandsbeheersysteem met mappenhiërarchie
\end{itemize}

\textbf{Voordelen:}
\begin{itemize}
    \item Minimale backend-vereisten
    \item Verbeterde schaalbaarheid
    \item Verminderde serverbelasting
\end{itemize}

\textbf{Nadelen:}
\begin{itemize}
    \item Complexere clientzijdige authenticatie
    \item Tijdelijke credentials vereisen vervangingslogica
    \item Beperkte bewerkingsmogelijkheden zonder backend
\end{itemize}

\subsubsection{Cross-Origin Resource Sharing (CORS) configuratie}

Voor directe communicatie tussen de Angular-applicatie en cloudopslagdiensten is correcte CORS-configuratie essentieel. De CORS-instellingen moeten worden geconfigureerd op de cloudopslagdienst om HTTP-verzoeken van de oorsprong van de Angular-applicatie toe te staan.

\subsection{Aanbevolen Angular UI-componenten en packages}

\subsubsection{ng-file-upload}

\texttt{ng-file-upload} is een veelzijdige Angular-package voor bestandsupload-functionaliteit.

\begin{listing}[H]
\begin{minted}{typescript}
import { FileUploadModule } from 'ng-file-upload';

@NgModule({
  imports: [
    FileUploadModule
  ]
})
export class AppModule { }
\end{minted}
\caption[Integratie van ng-file-upload]{Integratie van ng-file-upload.}
\end{listing}

\subsubsection{ngx-dropzone}

\texttt{ngx-dropzone} biedt een gebruiksvriendelijke drag-and-drop interface met uitstekende responsieve weergave:

\begin{listing}[H]
\begin{minted}{typescript}
import { NgxDropzoneModule } from 'ngx-dropzone';

@NgModule({
  imports: [
    NgxDropzoneModule
  ]
})
export class AppModule { }
\end{minted}
\caption[Integratie van ngx-dropzone]{Integratie van ngx-dropzone.}
\end{listing}

\subsubsection{Angular Material File Input}

Voor applicaties die Angular Material gebruiken, is een geïntegreerde bestandsinput component beschikbaar:

\begin{listing}[H]
\begin{minted}{typescript}
import { MatFileInputModule } from '@angular/material-components/file-input';

@NgModule({
  imports: [
    MatFileInputModule
  ]
})
export class AppModule { }
\end{minted}
\caption[Integratie van Angular Material File Input]{Integratie van Angular Material File Input.}
\end{listing}

\subsubsection{AWS SDK voor JavaScript/Angular}

Voor directe integratie met Amazon S3:

\begin{listing}[H]
\begin{minted}{typescript}
import { S3 } from 'aws-sdk';

export class S3UploadService {
  private s3: S3;
  
  constructor() {
    this.s3 = new S3({
      region: 'your-region',
      credentials: {
        accessKeyId: 'temporary-access-key',
        secretAccessKey: 'temporary-secret-key',
        sessionToken: 'temporary-session-token'
      }
    });
  }

  uploadFile(file: File, bucket: string, key: string): Observable<any> {
    // Implementatie voor directe upload
  }
}
\end{minted}
\caption[AWS SDK integratie]{Voorbeeld van directe S3-integratie met Angular.}
\end{listing}

\subsubsection{@azure/storage-blob}

Voor Microsoft Azure Blob Storage integratie:

\begin{listing}[H]
\begin{minted}{typescript}
import { BlobServiceClient } from '@azure/storage-blob';

export class AzureStorageService {
  private blobServiceClient: BlobServiceClient;
  
  constructor() {
    this.blobServiceClient = new BlobServiceClient(
      `https://accountname.blob.core.windows.net/sasToken`
    );
  }

  // Upload methodes
}
\end{minted}
\caption[Azure integratie]{Integratie met Azure Blob Storage via SAS-token.}
\end{listing}

\subsubsection{Bestandsverkenner componenten}

Voor het beheren van bestanden is een bestandsverkenner interface essentieel.

\textbf{Aanbevolen packages:}
\begin{itemize}
    \item \texttt{ngx-file-manager}: Volledig bestandsbeheersysteem met ondersteuning voor meerdere cloudproviders
    \item \texttt{angular-filemanager}: Aanpasbare component met CRUD-operaties voor cloudbestanden
\end{itemize}

\subsection{Integratiestrategie zonder backend-aanpassingen}

\begin{enumerate}
    \item \textbf{Implementeer een abstracte service laag}
\end{enumerate}

\begin{listing}[H]
\begin{minted}{typescript}
export abstract class CloudStorageService {
  abstract uploadFile(file: File, path: string): Observable<string>;
  abstract downloadFile(path: string): Observable<Blob>;
  abstract listFiles(directory: string): Observable<FileItem[]>;
  abstract deleteFile(path: string): Observable<boolean>;
}
\end{minted}
\caption[Abstracte storage-interface]{Abstracte interface voor cloudopslag.}
\end{listing}

\begin{enumerate}
\setcounter{enumi}{1}
    \item Maak provider-specifieke implementaties
    \item Gebruik een factory pattern voor dynamische service-injectie
    \item Ontwerp gedeelde UI-componenten die met deze service werken
\end{enumerate}

\begin{listing}[H]
\begin{minted}{typescript}
@Component({
  selector: 'app-cloud-uploader',
  template: `
    <ngx-dropzone (change)="onFileSelect($event)">
      <ngx-dropzone-label>Sleep bestanden hierheen</ngx-dropzone-label>
      <ngx-dropzone-preview *ngFor="let file of files" 
                            [removable]="true" 
                            (removed)="onRemove(file)">
        <ngx-dropzone-label>{{ file.name }}</ngx-dropzone-label>
      </ngx-dropzone-preview>
    </ngx-dropzone>
    <mat-progress-bar *ngIf="uploading" [value]="uploadProgress"></mat-progress-bar>
  `
})
export class CloudUploaderComponent {
  @Input() storageService: CloudStorageService;
  // Implementatie
}
\end{minted}
\caption[Herbruikbare uploader]{Voorbeeld van herbruikbare cloud uploader component.}
\end{listing}

\subsection{Beveiligingsoverwegingen}

Bij directe cloudintegratie zijn er belangrijke beveiligingsaspecten:
\begin{itemize}
    \item \textbf{Tijdelijke credentials:} Gebruik korte-termijn tokens via een minimale backend-service
    \item \textbf{Bestandsvalidatie:} Implementeer clientzijdige validatie voor bestandsgrootte en bestandstype
    \item \textbf{Toegangscontrole:} Beperk gebruikerstoegang tot specifieke paden of containers
\end{itemize}

\subsection{Vergelijking van cloudopslagdienst-integratie}

\begin{table}[H]
    \centering
    \footnotesize
    \begin{tabular}{lcr}
      \toprule
      \textbf{Cloudopslagdienst} & \textbf{Integratie en package} & \textbf{Beperkingen} \\
      \midrule
      Amazon S3 & Hoog (presigned URLs, STS) – \texttt{aws-sdk} & Complexere configuratie \\
      DigitalOcean Spaces & Hoog (S3-compatibel) – \texttt{aws-sdk} + custom endpoint & Beperkte functieset \\
      Nextcloud & Middel (WebDAV) – \texttt{nextcloud-client} & Vereist eigen implementatie \\
      Microsoft Azure & Hoog (SAS tokens) – \texttt{@azure/storage-blob} & Eenvoudige setup \\
      Google Cloud Storage & Hoog (signed URLs) – \texttt{@google-cloud/storage} & Veel mogelijkheden \\
      \bottomrule
    \end{tabular}
    \caption[Vergelijking cloudintegraties]{\label{tab:cloud-vergelijking}Vergelijking van integratiemethoden voor cloudopslagdiensten.}
 \end{table}

\subsection{Conclusie}

De integratie van diverse cloudopslagdiensten in Angular-applicaties zonder uitgebreide backend-aanpassingen is mogelijk door gebruik te maken van specifieke UI-componenten zoals \texttt{ngx-dropzone}, \texttt{ng-file-upload}, en cloudspecifieke SDK's. Het implementeren van een abstracte service laag met provider-specifieke implementaties biedt een flexibele en onderhoudbare architectuur. Door gebruik te maken van pre-signed URLs of STS-tokens kan directe communicatie met cloudopslagdiensten worden gerealiseerd, terwijl de veiligheid gewaarborgd blijft.

Door de juiste combinatie van UI-componenten, authenticatiestrategieën en abstracte interfaces te kiezen, kan een robuuste oplossing worden gecreëerd die meerdere cloudopslagdiensten ondersteunt zonder significante backend-ontwikkeling. Dit resulteert in een schaalbare en efficiënte applicatiearchitectuur die directe cloudintegratie mogelijk maakt.
