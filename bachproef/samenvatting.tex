%%=============================================================================
%% Samenvatting
%%=============================================================================

% TODO: De "abstract" of samenvatting is een kernachtige (~ 1 blz. voor een
% thesis) synthese van het document.
%
% Een goede abstract biedt een kernachtig antwoord op volgende vragen:
%
% 1. Waarover gaat de bachelorproef?
% 2. Waarom heb je er over geschreven?
% 3. Hoe heb je het onderzoek uitgevoerd?
% 4. Wat waren de resultaten? Wat blijkt uit je onderzoek?
% 5. Wat betekenen je resultaten? Wat is de relevantie voor het werkveld?
%
% Daarom bestaat een abstract uit volgende componenten:
%
% - inleiding + kaderen thema
% - probleemstelling
% - (centrale) onderzoeksvraag
% - onderzoeksdoelstelling
% - methodologie
% - resultaten (beperk tot de belangrijkste, relevant voor de onderzoeksvraag)
% - conclusies, aanbevelingen, beperkingen
%
% LET OP! Een samenvatting is GEEN voorwoord!

%%---------- Nederlandse samenvatting -----------------------------------------
%
% TODO: Als je je bachelorproef in het Engels schrijft, moet je eerst een
% Nederlandse samenvatting invoegen. Haal daarvoor onderstaande code uit
% commentaar.
% Wie zijn bachelorproef in het Nederlands schrijft, kan dit negeren, de inhoud
% wordt niet in het document ingevoegd.

\IfLanguageName{english}{%
\selectlanguage{dutch}
\chapter*{Samenvatting}
\lipsum[1-4]
\selectlanguage{english}
}{}

%%---------- Samenvatting -----------------------------------------------------
% De samenvatting in de hoofdtaal van het document

\chapter*{\IfLanguageName{dutch}{Samenvatting}{Abstract}}

Aquarius Zwemclub Lebbeke (AZL) maakt momenteel gebruik van Dropbox voor het delen van bestanden tussen lesgevers. Deze oplossing kent echter meerdere beperkingen, 
waaronder verouderde toegangsrechten, gebrekkige integratie met bestaande systemen en een lage gebruiksvriendelijkheid voor lesgevers zonder Dropbox-account. 
Om deze problemen aan te pakken, werd in deze bachelorproef gezocht naar een alternatieve cloudopslagoplossing die beter inspeelt op de noden van AZL.

De centrale onderzoeksvraag luidt: \textbf{Hoe kan AZL een cloudopslagoplossing implemente- ren die zorgt voor een beter beheer van de toe- gang tot bestanden, eenvoudig integreert met de bestaande systemen, en gebruiksvriendelijk is voor alle lesgevers, zonder dat zij hiervoor een extra account hoeven aan te maken?}

Om tot een antwoord te komen werd een gestructureerde methode toegepast. Eerst werd via een literatuurstudie een longlist van mogelijke oplossingen samengesteld. 
Vervolgens werden de functionele en niet-functionele vereisten van AZL in kaart gebracht, waarop een shortlist van de meest geschikte alternatieven werd opgesteld. 
Hieruit werd Nextcloud geselecteerd als meest veelbelovende optie. Ter evaluatie werd een proof of concept (PoC) ontwikkeld en getest binnen de 
bestaande IT-infrastructuur van AZL.

De resultaten tonen aan dat Nextcloud door zijn open-source karakter, flexibiliteit en uitgebreide mogelijkheden voor toegangsbeheer een sterke match vormt voor AZL. 
De oplossing zorgde voor een vereenvoudigd beheer van toegangsrechten, verbeterde integratie met de bestaande webapplicatie en een gebruiksvriendelijke ervaring voor 
de lesgevers, zonder dat zij een extra account hoefden aan te maken.

Op basis van dit onderzoek kan geconcludeerd worden dat Nextcloud een geschikte en toekomstbestendige oplossing is voor AZL. De implementatie ervan biedt niet 
alleen technische voordelen, maar draagt ook bij aan een efficiëntere werking en vermindert de administratieve last. Voor andere verenigingen met vergelijkbare 
noden kan dit onderzoek dienen als leidraad bij de keuze voor een geschikte cloudopslagoplossing.