\chapter{\IfLanguageName{dutch}{Longlist}{Longlist}}
\label{ch:longlist}

Op basis van de literatuurstudie (Hoofdstuk~\ref{ch:stand-van-zaken}) en de initiële requirementsanalyse (Hoofdstuk~\ref{ch:requirement-analyse}) werd een longlist van potentiële cloudopslagoplossingen samengesteld. Het doel van deze fase was om een breed scala aan opties te identificeren die potentieel zouden kunnen voldoen aan de basisbehoeften van Aquarius Zwemclub Lebbeke (AZL), zonder in dit stadium al diepgaand te filteren. Zowel commerciële als open-source alternatieven werden overwogen.

De volgende cloudopslagoplossingen werden opgenomen in de longlist voor verdere evaluatie:

\begin{itemize}
    \item \textbf{Dropbox:} De huidige oplossing die AZL gebruikt, meegenomen als benchmark en voor directe vergelijking met alternatieven. Ondanks de gekende problemen (zie Hoofdstuk 1 en 4), biedt het een bekende interface.
    \item \textbf{Amazon S3 (Simple Storage Service):} Een toonaangevende object storage dienst, gekend om zijn schaalbaarheid, betrouwbaarheid en uitgebreide functionaliteiten en API's.
    \item \textbf{DigitalOcean Spaces:} Een S3-compatibele object storage dienst, specifiek gericht op ontwikkelaars en kleinere organisaties, met een focus op eenvoud en voorspelbare prijzen. Relevant gezien AZL's bestaande infrastructuur bij DigitalOcean.
    \item \textbf{Nextcloud:} Een populaire open-source oplossing die self-hosting of gehoste varianten toelaat, met sterke focus op controle over data, privacy en uitbreidbaarheid.
    \item \textbf{Microsoft Azure Blob Storage:} De object storage oplossing binnen het Microsoft Azure ecosysteem, gekend om zijn integratie met andere Azure diensten en enterprise-level features.
    \item \textbf{Google Cloud Storage:} De object storage dienst van Google Cloud Platform, die schaalbaarheid, prestaties en integratie met het Google ecosysteem biedt.
\end{itemize}

Deze lijst vormde het startpunt voor de volgende fase: het opstellen van een shortlist door deze oplossingen grondiger te toetsen aan de specifieke, geprioriteerde vereisten van AZL. Criteria zoals gebruiksvriendelijkheid voor lesgevers, efficiënt toegangsbeheer, integratiemogelijkheden met de bestaande Angular/Node.js applicatie en WordPress site, beveiliging, onderhoudbaarheid en kosten speelden hierbij een centrale rol.
