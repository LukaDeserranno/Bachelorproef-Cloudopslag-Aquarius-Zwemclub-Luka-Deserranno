%%=============================================================================
%% Inleiding
%%=============================================================================

\chapter{\IfLanguageName{dutch}{Inleiding}{Introduction}}%
\label{ch:inleiding}

Binnen sportverenigingen speelt een efficiënte digitale werking een steeds belangrijkere rol. Documenten zoals lesplanningen, presentaties en evaluaties dienen vlot gedeeld en beheerd te worden om de dagelijkse werking te ondersteunen. Bij Aquarius Zwemclub Lebbeke (AZL) wordt hiervoor momenteel gebruikgemaakt van Dropbox. Hoewel dit platform aanvankelijk voldeed aan de basisbehoeften, ondervindt AZL vandaag structurele problemen zoals verouderde toegangsrechten, het ontbreken van integratie met bestaande systemen, en beperkte toegankelijkheid voor lesgevers zonder Dropbox-account. Deze uitdagingen zorgen voor een verhoogde administratieve werklast en verhinderen een efficiënte samenwerking.

AZL beschikt over een bestaande IT-infrastructuur die onder andere bestaat uit een WordPress-website, een op maat ontwikkelde webapplicatie en hosting via DigitalOcean. De afwezigheid van een geïntegreerde en centraal beheerde cloudopslagoplossing leidt tot extra manueel werk bij het beheren van gebruikers en hun toegangsrechten. Bovendien missen lesgevers vaak de juiste middelen om zonder drempels toegang te krijgen tot de noodzakelijke documenten. 

Het doel van deze bachelorproef is om een duurzame, veilige en gebruiksvriendelijke cloudopslagoplossing te selecteren die niet alleen aansluit bij de specifieke noden van AZL, maar ook eenvoudig te integreren is met de bestaande infrastructuur. Hierbij wordt bijzondere aandacht besteed aan kostenbeheersing, schaalbaarheid en onderhoudsvriendelijkheid. Uiteindelijk wordt de gekozen oplossing geïmplementeerd in een proof of concept (PoC), waarmee getest wordt of deze effectief bijdraagt aan een vlottere werking van AZL.

\section{Probleemstelling}%
\label{sec:probleemstelling}

Aquarius Zwemclub Lebbeke is een lokale sportvereniging die afhankelijk is van digitale bestanden voor het ondersteunen van haar lessen en trainingen. Het huidige gebruik van Dropbox biedt niet langer een toekomstgerichte oplossing. Toegangsrechten worden niet automatisch bijgewerkt, integratie met bestaande toepassingen ontbreekt, en niet elke lesgever beschikt over een Dropbox-account. Dit alles leidt tot een verhoogde werkdruk voor de beheerders en zorgt ervoor dat niet alle lesgevers even eenvoudig toegang hebben tot de benodigde documenten.

Deze bachelorproef richt zich daarom exclusief op AZL en heeft als doel een cloudopslagoplossing te onderzoeken die eenvoudig te beheren is, goed integreert met de huidige infrastructuur en gebruiksvriendelijk is voor alle betrokkenen. De doelgroep bestaat uit de lesgevers en beheerders van AZL, met als belangrijkste belanghebbende de webmaster van de club die instaat voor het technisch beheer van de systemen.

\section{Onderzoeksvraag}%
\label{sec:onderzoeksvraag}

De centrale onderzoeksvraag van deze bachelorproef luidt:

\begin{quote}
Welke cloudopslagoplossing biedt Aquarius Zwemclub Lebbeke de beste balans tussen gebruiksvriendelijkheid, kostenbeheersing, integratie met bestaande systemen en efficiënt gebruikersbeheer?
\end{quote}

Om deze hoofdvraag te beantwoorden, worden volgende deelvragen onderzocht:

\begin{itemize}
  \item Welke cloudopslagoplossingen zijn geschikt voor een kleine sportvereniging zoals AZL?
  \item Wat zijn de voor- en nadelen van de verschillende beschikbare oplossingen?
  \item Hoe kan een cloudopslagoplossing geïntegreerd worden met de bestaande systemen van AZL?
  \item Hoe kan het toegangsbeheer efficiënt, veilig en up-to-date worden gehouden?
  \item Wat zijn de kosten van de verschillende oplossingen en hoe verhouden deze zich tot het beschikbare budget van AZL?
\end{itemize}

\section{Onderzoeksdoelstelling}%
\label{sec:onderzoeksdoelstelling}

Deze bachelorproef heeft als doel het selecteren, implementeren en evalueren van een geschikte cloudopslagoplossing voor AZL. Het einddoel is om een proof of concept (PoC) te ontwikkelen waarin de gekozen oplossing geïntegreerd wordt in de bestaande IT-omgeving van AZL. Het succes van het onderzoek wordt bepaald aan de hand van de volgende criteria:

\begin{itemize}
  \item De oplossing biedt eenvoudig beheer van gebruikers en toegangsrechten.
  \item De oplossing integreert met de bestaande WordPress-website en webapplicatie van AZL.
  \item Lesgevers kunnen zonder bijkomende accounts vlot toegang krijgen tot de documenten.
  \item De oplossing blijft financieel haalbaar binnen het beperkte budget van AZL.
\end{itemize}

Het resultaat van deze bachelorproef bestaat uit een werkend prototype van de gekozen oplossing, een grondige evaluatie ervan binnen AZL en aanbevelingen voor een mogelijke volledige implementatie.

\section{\IfLanguageName{dutch}{Opzet van deze bachelorproef}{Structure of this bachelor thesis}}%
\label{sec:opzet-bachelorproef}

% Het is gebruikelijk aan het einde van de inleiding een overzicht te
% geven van de opbouw van de rest van de tekst. Deze sectie bevat al een aanzet
% die je kan aanvullen/aanpassen in functie van je eigen tekst.

De rest van deze bachelorproef is als volgt opgebouwd:
In Hoofdstuk~\ref{ch:stand-van-zaken} wordt de stand van zaken binnen het onderzoeksdomein besproken aan de hand van een literatuurstudie. Hierbij wordt relevante bestaande kennis in kaart gebracht.
Hoofdstuk~\ref{ch:methodologie} licht de gehanteerde methodologie toe en bespreekt de onderzoeksfasen en -technieken die werden gebruikt om de onderzoeksvragen te beantwoorden.
Vervolgens beschrijft Hoofdstuk~\ref{ch:vergelijking} het selectieproces: het start met het definiëren van de functionele en niet-functionele vereisten voor AZL, presenteert een overzicht van potentiële cloudopslagalternatieven, vergelijkt deze aan de hand van de vereisten, en motiveert de uiteindelijke selectie van de oplossing.
In Hoofdstuk~\ref{ch:proof-of-concept} wordt de Proof of Concept (PoC) van de geselecteerde oplossing opgebouwd en geëvalueerd, met aandacht voor de praktische implementatie en integratie binnen de context van AZL.
Tot slot bevat Hoofdstuk~\ref{ch:conclusie} de conclusies van dit onderzoek. Er wordt een antwoord geformuleerd op de onderzoeksvragen en er worden aanbevelingen gedaan voor de verdere implementatie bij AZL en mogelijk toekomstig onderzoek.