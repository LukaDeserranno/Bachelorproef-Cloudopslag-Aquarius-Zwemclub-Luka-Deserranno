%%=============================================================================
%% Voorwoord
%%=============================================================================

\chapter*{\IfLanguageName{dutch}{Woord vooraf}{Preface}}%
\label{ch:voorwoord}

Deze bachelorproef markeert het eindpunt van mijn opleiding Toegepaste Informatica en is tegelijk een weerspiegeling van mijn interesse in digitale efficiëntie binnen kleinschalige organisaties. Vanuit mijn persoonlijke betrokkenheid bij sportverenigingen merkte ik dat technologie vaak beperkt of omslachtig wordt ingezet. Toen ik hoorde over de uitdagingen bij Aquarius Zwemclub Lebbeke rond het delen van documenten, zag ik meteen een kans om met een praktische en duurzame oplossing aan de slag te gaan.

De keuze om een eigen toegangsbeheersysteem uit te werken in combinatie met een cloudopslagdienst was technisch uitdagend, maar bijzonder leerrijk. Vooral het opzetten van fijnmazige toegangsrechten en de integratie met een gebruiksvriendelijke backendstructuur hebben mijn inzicht in softwarearchitectuur sterk verdiept.

Ik wil graag enkele mensen bedanken die mij doorheen dit traject hebben ondersteund. In de eerste plaats mijn promotor Mevrouw S. VanderMeersch, voor haar constructieve begeleiding en waardevolle feedback. Ook veel dank aan Dhr. Thomas Aelbrecht, die niet alleen als co-promotor, maar ook als webmaster van AZL een onmisbare rol speelde in het aanleveren van context, technische vereisten en het testen van de oplossing in de praktijk.

Verder gaat mijn dank uit naar de lesgevers en medewerkers van AZL die meegewerkt hebben aan de evaluatie van het systeem. Tot slot wil ik ook mijn familie en vrienden bedanken voor hun aanmoediging en geduld tijdens de vele uren achter mijn scherm.

Luka Deserranno \\
Academiejaar 2024–2025
